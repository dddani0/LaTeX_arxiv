\documentclass{article}
%%%%%%%%%%%%%%%%%%%%%%%%%%%%%%%%%%%%%%%%%%%%%%%%%%%%%%%%%%%%%%%
\usepackage{hhline}
%%%%%%%%%%%%%%%%%%%%%%%%%%%%%%%%%%%%%%%%%%%%%%%%%%%%%%%%%%%%%%%
\begin{document}
\begin{table}[]
    \caption{A complex example for handling tables}
    \centering
    \renewcommand{\caption}[1]{#1}
   \caption{Coefficients of friction} \\
    \begin{tabular}{||c|l|c|c|l||}
     \hhline{|t:=====:t|}
        \multicolumn{2}{|c|}{Static friction} & 3 & \multicolumn{2}{|c|}{Sliding friction} \\
        \hhline{|:-----:|}
        dry and clean  & lubricated &  & dry and clean & lubricated  \\
        \hhline{|:=====:|}
        0.15 & 0.111 & steel on steel & 0.14 & 0.001  \\
        \hhline{||}
        0.19  & 0.1 & steel on iron & 0.18 & 0.001 \\
        \hhline{|b:=====:b|}
    \end{tabular}
    \label{tab:my_label}
\end{table}
\par
As it is usual, type your own name here, with all the special symbols used in your language: Mikus Dániel. \par
Table 1. shows loads of examples on how to deal with tables. \par
If any tables are present, it is possible to create a “collection of the table captions”:
\listoftables
\end{document}