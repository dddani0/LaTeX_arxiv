\documentclass[12pt]{article}
%%%%%%%%%%%%%%%%%%%%%%%%%%%%%%%%%%%%%%%%%%%%%%%%%%%%%%%%%
% Packages (preamble)
%%%%%%%%%%%%%%%%%%%%%%%%%%%%%%%%%%%%%%%%%%%%%%%%%%%%%%%%%
%%%%%%%%%%%%%%%%%%%%%%%%%%%%%%%%%%%%%%%%%%%%%%%%%%%%%%%%%
\usepackage{glossaries}
\usepackage{amsmath}
%%%%%%%%%%%%%%%%%%%%%%%%%%%%%%%%%%%%%%%%%%%%%%%%%%%%%%%%%
\makeglossaries
%%%%%%%%%%%%%%%%%%%%%%%%%%%%%%%%%%%%%%%%%%%%%%%%%%%%%%%%%
\newcommand{\makeboxlabel}[1]{\fbox{!!!}\hfill}% \hfill fills the label box
%%%%%%%%%%%%%%%%%%%%%%%%%%%%%%%%%%%%%%%%%%%%%%%%%%%%%%%%%
\newglossaryentry{dictionary}
{
        name=Dictionaries:
}
\newglossaryentry{lawofmaxwell}
{
        name=\text{4}^{\text{th}} \text{law of Maxwell:}
}
%%%%%%%%%%%%%%%%%%%%%%%%%%%%%%%%%%%%%%%%%%%%%%%%%%%%%%%%%
\begin{document}
As a first exercise, type your own name here, with all the special symbols used in your language: Mikus Dániel \par
Let us focus on the usage of lists.
\begin{enumerate}
    \item This is a numbered list
    \begin{enumerate}
        \item Different levels are distinguished by different numbering in ~\LaTeX
        \item[(f)] Now I changed the corresponding counter of ~\LaTeX to 5 so the sign of this item is f which is the 6th character in the alphabet.
    \end{enumerate}
\end{enumerate}
\begin{itemize}
    \item The same can't be told about ...
    \begin{itemize}
        \item ...unnumbered lists.
    \end{itemize}
    \begin{makeboxlabel}
        \item But, as you see, the label of an item can be changed freely.
    \end{makeboxlabel}
\end{itemize} \par
\textbf{\gls{dictionary}} In dictionaries we use a third type of list in which a word or a phrase plays the role of the label of an item. \\ \\
\textbf{\gls{lawofmaxwell}} \\ \\
\centering
\begin{equation}
    \oint \B \cdot ds 
\end{equation}
\end{document}
%%%%%%%%%%%%%%%%%%%%%%%%%%%%%%%%%%%%%%%%%%%%%%%%%%%%%%%%%