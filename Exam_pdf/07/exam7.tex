\documentclass[12pt]{article}
%%%%%%%%%%%%%%%%%%%%%%%%%%%%%%%%%%%%%%%%%%%%%%%%%%%%%%%%%
% Packages (preamble)
%%%%%%%%%%%%%%%%%%%%%%%%%%%%%%%%%%%%%%%%%%%%%%%%%%%%%%%%%
%%%%%%%%%%%%%%%%%%%%%%%%%%%%%%%%%%%%%%%%%%%%%%%%%%%%%%%%%

%%%%%%%%%%%%%%%%%%%%%%%%%%%%%%%%%%%%%%%%%%%%%%%%%%%%%%%%%
\begin{document}
As the first exercise, type your own name here, with all the special symbols used in your language Mikus Dániel.
\begin{enumerate}
\item This is a numbered list.
\begin{enumerate}
\item Different levels are distinguished by different numbering in \LaTeX.
\setcounter{enumii}{5}
\item Now I changed the corresponding counter of \LaTeX to 5 so the sign of this item is f which is the 6\textsuperscript{th} character in the alphabet.
\end{enumerate}
\end{enumerate}

\begin{itemize}
\item The same can be told about \dots
\begin{itemize}
\item \dots unnumbered lists.
\item[\fbox{!!!}] But, as you see, the label of an item can be changed freely. \footnote{Note that the boxed (frame) exclamation marks the plasy the role of the label of that item.}
\end{itemize}
\end{itemize}

\begin{description}
\item[Dictionaries:] In dictionaries we use a third type of list in which a word or a phrase plays the role of the label of an item.
\item[4\textsuperscript{th} law of Maxwell:]
\begin{equation}
\oint_g \mathbf{B}\cdot d\mathbb{s}=\mu_0\int_A\left(\mathbf{J}+\epsilon_0\frac{\partial\mathbf{E}}{\partial t}\right)\cdot d\mathbf{A}
\end{equation}
\end{description}
\end{document}
%%%%%%%%%%%%%%%%%%%%%%%%%%%%%%%%%%%%%%%%%%%%%%%%%%%%%%%%%