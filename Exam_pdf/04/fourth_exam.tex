\documentclass[12pt]{article}
%%%%%%%%%%%%%%%%%%%%%%%%%%%%%%%%%%%%%%%%%%%%%%%%%%%%%%%%%
\usepackage{marginnote}
\usepackage{amsmath} %\normal miatt
\usepackage{pagenote} % endnotes miatt
%%%%%%%%%%%%%%%%%%%%%%%%%%%%%%%%%%%%%%%%%%%%%%%%%%%%%%%%%
\renewcommand*{\notedivision}{\section*{\notesname}}
\renewcommand*{\pagenotesubhead}[1]{}
\renewcommand{\notesname}{End Notes}
\renewcommand\notenuminnotes[1]{{$^#1$}}
\makepagenote % endnote megjelenítése
%%%%%%%%%%%%%%%%%%%%%%%%%%%%%%%%%%%%%%%%%%%%%%%%%%%%%%%%%
\begin{document}
\par
You first task is to type your own name here, with all the special symbols used in your language: Mikus Dániel. \par
At the end of this document you can find a References section; I refer to its elements: \cite{Lotr, lmp}.  \par
While editing a document, you may find that you need to typeset something as a footnote \footnote{...like this one...} or as a "margin-note" \marginnote{as this one} or even as a note at the end of as this your document \pagenote{Now, this is a note at the end of the document.} For the latter one you will need to use a package. \par
And, as usual, you can find a beautiful equation (\ref{eq:1}) on page 1 (use correct cross-referencing):
\begin{equation} \label{eq:1}
    \left( \exists 0 \leq p_k \leq 1, | \psi _k \rangle \text{~is separable,~} \sum_k p_k = 1 \right)~ \rho = \sum_k p_k | \psi \rangle \langle \psi _k |
\end{equation}
\bibliographystyle{acm}
\bibliography{References}
\begin{thebibliography}{9}
\bibitem{Lotr}
J.R.R. Tolkien: 
\textit{The Lord of the Rings}. Mariner Books; Anniversary Edition
Edition (2012).
\bibitem{lmp}
The Hardvard Lampoon:
\textit{Bored of the Rings: A parody.} Touchstone 
(2012) ISBN-10: 1451672667
\end{thebibliography}
\printnotes*
\end{document}