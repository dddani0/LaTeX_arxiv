\documentclass{article}
%%%%%%%%%%%%%%%%%%%%%%%%%%%%%%%%%%%%%%%%%%
%                   ┌┐  ┌┐
%                   ││  ││
%┌──┐┌─┐┌──┐┌──┐┌┐┌┐│└─┐││ ┌──┐
%│┌┐││┌┘││─┤│┌┐││└┘││┌┐│││ ││─┤
%│└┘│││ ││─┤│┌┐││││││└┘││└┐││─┤
%│┌─┘└┘ └──┘└┘└┘└┴┴┘└──┘└─┘└──┘
%││
%└┘
%%%%%%%%%%%%%%%%%%%%%%%%%%%%%%%%%%%%%%%%%%
\usepackage{amsmath}
%%%%%%%%%%%%%%%%%%%%%%%%%%%%%%%%%%%%%%%%%%
\usepackage{multicol} % multiple column package
%%%%%%%%%%%%%%%%%%%%%%%%%%%%%%%%%%%%%%%%%%
\usepackage{authblk}
%%%%%%%%%%%%%%%%%%%%%%%%%%%%%%%%%%%%%%%%%%
\usepackage{blindtext} %Remove later
%%%%%%%%%%%%%%%%%%%%%%%%%%%%%%%%%%%%%%%%%%
\usepackage{indentfirst}
%%%%%%%%%%%%%%%%%%%%%%%%%%%%%%%%%%%%%%%%%%
\usepackage{lmodern}
\usepackage{etoolbox}
%%%%%%%%%%%%%%%%%%%%%%%%%%%%%%%%%%%%%%%%%%
\usepackage{fancyhdr} 
\pagestyle{fancy}
\fancyhf{}
\renewcommand{\headrulewidth}{0pt} % sets both header and footer to nothing
\fancyhead[R]{\thepage}
\usepackage{authblk} % több author címke
%%%%%%%%%%%%%%%%%%%%%%%%%%%%%%%%%%%%%%%%%%
% Roman numerals for numbering
\renewcommand{\thesection}{\Roman{section}} 
%%%%%%%%%%%%%%%%%%%%%%%%%%%%%%%%%%%%%%%%%%
%┌──┐ ┌───┐┌───┐┌┐  ┌┐
%│┌┐│ │┌─┐│└┐┌┐││└┐┌┘│
%│└┘└┐││ ││ ││││└┐└┘┌┘
%│┌─┐│││ ││ ││││ └┐┌┘
%│└─┘││└─┘│┌┘└┘│  ││
%└───┘└───┘└───┘  └┘
%%%%%%%%%%%%%%%%%%%%%%%%%%%%%%%%%%%%%%%%%%
\makeatletter
\title{
\patchcmd{\@maketitle}{\large \lineskip}{\Large \lineskip}{\typeout{OK 2}}{\typeout{Failed 2}}
\textbf
{Fractional Marcus-Hush-Chidsey-Yakopcic current-voltage model for redox-based resistive memory devices}
}
\makeatother
\author[1]{Georgii Paradezhenko}
\author[1]{Dmitrii Prodan}
\author[1]{Anastasiia Pervishko}
\author[1]{Dmitry Yudin}
\author[2, 3]{Anis Allagui}
\affil[1]{\textit{Skolkovo Institute of Science and Technology, Moscow 121205, Russia}}
\affil[2]{\textit{Department of Sustainable and Renewable Energy Engineering, University of Sharjah, Sharjah, P.O. Box 27272, United Arab Emirates}}
\affil[3]{\textit{Department of Mechanical and Materials Engineering,
Florida International University, Miami, FL33174, United States}}
\renewcommand{\abstractname}{}    % clear the title
\setlength{\parindent}{20pt}
%%%%%%%%%%%%%%%%%%%%%%%%%%%%%%%%%%%%%%%%%%
%%%%%%%%%%%%%%%%%%%%%%%%%%%%%%%%%%%%%%%%%%
%%               BODY
%%%%%%%%%%%%%%%%%%%%%%%%%%%%%%%%%%%%%%%%%%
%%%%%%%%%%%%%%%%%%%%%%%%%%%%%%%%%%%%%%%%%%
\begin{document}

%%%%%%%%%%%%%%%%%%%%%%%%%%%%%%%%%%%%%%%%%%
% Showcase title on the page
\maketitle
%%%%%%%%%%%%%%%%%%%%%%%%%%%%%%%%%%%%%%%%%%

%%%%%%%%%%%%%%%%%%%%%%%%%%%%%%%%%%%%%%%%%%
% Skip first-page numbering
\thispagestyle{empty}
%%%%%%%%%%%%%%%%%%%%%%%%%%%%%%%%%%%%%%%%%%

%%%%%%%%%%%%%%%%%%%%%%%%%%%%%%%%%%%%%%%%%%
% Change the abstract size
{\renewenvironment{quotation}%
               {\list{}{\addtolength{\leftmargin}{2em} % change this value to add or remove length to the the default
                        \listparindent 1.5em%
                        \itemindent    \listparindent%
                        \rightmargin   \leftmargin%
                        \parsep        \z \plus \p}%
                \item\relax}%
               {\endlist}%
%%%%%%%%%%%%%%%%%%%%%%%%%%%%%%%%%%%%%%%%%%
\begin{abstract}
We propose a circuit-level model combining the Marcus-Hush-Chidsey electron current equation and the Yakopcic equation for the state variable for describing resistive switching memory devices of the structure metal–ionic conductor–metal. We extend the dynamics of the state variable originally described by a first-order time derivative by introducing a fractional derivative with an arbitrary order between zero and one. We show that the extended model fits with great fidelity the current-voltage characteristic data obtained on a Si electrochemical metallization memory device with Ag-Cu alloy\\
\end{abstract}
%%%%%%%%%%%%%%%%%%%%%%%%%%%%%%%%%%%%%%%%%%
% Set space between two columns
\setlength{\columnsep}{0.8cm}
%%%%%%%%%%%%%%%%%%%%%%%%%%%%%%%%%%%%%%%%%%
\begin{multicols}{2}
{\centering %Only center the section title
\section{Introduction}
}
Substantial research efforts have been dedicated to the
development of electrically-controlled resistive switching in metal-insulator-metal (MIM) devices or memristors, going from new materials discovery
to modelling and simulation, and design and applications.
With both memory and logic capabilities combined at
the hardware level, in addition to long retention times
and high switching rates at relatively low energy
consumption, these devices are favorably seen as the
next-generation building blocks for nonvolatile memories and neuromorphic computing applications. In
a typical memristor, the resistive switching is based on
the electrically-stimulated change of cell resistance usually driven by internal ion redistribution, which actually depends not only on the applied excitation but also
on the past history of the excitation
. Physical mechanisms associated with these reversible transitions have
been attributed to different effects including valencechange 16, electrochemical metallization17, and phase
change effects18. They can be either abrupt (binary)
or gradual (analogue), and evolve at different timescales,
leading to rich and complex device behaviors in this seemingly simple device structure of just three layers. Furthermore, with the wide range of diversity in memristors
materials and their morphologies, operating mechanisms,
and manufacturing technologies there is an urgent need
for the development of a general model capable of capturing accurately and effectively their complex nonlinear
dynamics. This is crucial not only for the characterization and comparison between different memristor devices,
but also for the investigation of larger scale memristorbased circuits and hybrid hardware architectures, and
also to explore similar behaviors observed for instance in
biological synapse systems
\par
While models at different size scales and thus with different degrees of physical details and computational complexity have been developed for memritors, including but not limited to ab initio, Kinetic Monte Carlo, and finite element method modells, in this work we focus on the circuit-level (compact) current-voltage behavior of the memristors. From this point of view, Memristors are generally described by the systems of coupled equations: 
\begin{align}{}
     i &= G(v,x)v, \\
     \dot{x} &= f(x,v),
\end{align}

where $i = i(t)$ is the current through the device, $v = v(t)$
is the applied voltage, and $x = x(t)$ corresponds to a state
variable or a group of state variables that quantify the
internal dynamics of the device. These are, for example,
width of doping region, concentration of vacancies in the gap region, and tunneling barrier width. State vairables can not be observed from external electrical behavior. Eq1. (1) follows the $i-v$ curve of the resistive device in the question with $G(v,x)$ being the generalized conductance, whereas Eq. describes the dynamics of the it's internal state x based on it's prehistory. The actual stat of a memristor can only be determined by solving Eqs. (1) and (2) self-consistently. Memristive systems as featured
in terms of Eqs. (1) and (2) are known to possess a pinched hysteresis loop at the origin in the i-v plane in the response to any periodic voltage source. \par
Being versatile and modular enough it is the Yakopcic model 27–29 which is most often used to simulate
the nonlinear i-v characteristic of wide range of memristors in response to sinusoidal and repetitive sweeping
inputs. The model takes into account electron transmission effects, voltage threshold for state variable motion,
and nonlinear velocity function for oxygen vacancies or
dopant drift, considered to be the most relevant internal
state information29. It follows on the steps of Strukov et
al. work30, and describes the memristor as two resistors
in series characterized by electron transmission equations
so that:
\begin{align}{}
     i(t) = h_1 (v)x + h_2 (v)(1-x)
\end{align}
Here, h1 is used to model the behavior in the low resistance state of the device, and h2 captures its behavior in the high-resistance state.
The two electron transmission equations are weighted and mixed by the state
variable x which is set to take values between zero and
one25. In memristive devices, it is the rate of change of
the state variable x that is explicitly determined (2), and
is given in the Yakopcic memristor model by the product
of the two composite functions g(v) and f(x) such that29:
\begin{align}{}
     \dot{x} = g(v)f(x).
\end{align}
\end{multicols}
\end{document}