\documentclass{article}
\usepackage[top=0cm, bottom=5cm, outer=2cm, inner=2cm, marginparwidth=5cm, marginparsep=5cm]{geometry}
\usepackage{multicol}
\usepackage{authblk}
\usepackage{blindtext}
\title{\textbf{Fractional Marcus-Hush-Chidsey-Yakopcic current-voltage model for redox-based\\ resistive memory devices}}

\author[1]{Georgii Paradezhenko}
\author[1]{Dmitrii Prodan}
\author[1]{Anastasiia Pervishko}
\author[1]{Dmitry Yudin}
\author[2]{Anis Allagui}
\affil[1]{\textit{Skolkovo Institute of Science and Technology, Moscow 121205, Russia}}
\affil[2]{\textit{Department of Sustainable and Renewable Energy Engineering, University of Sharjah, Sharjah, P.O. Box 27272, United Arab Emirates}}
\affil[3]{\textit{Department of Mechanical and Materials Engineering,
Florida International University, Miami, FL33174, United States}}
\renewcommand{\abstractname}{}    % clear the title
\renewcommand{\absnamepos}{empty} % originally center
\begin{document}

\maketitle

\begin{abstract}
    We propose a circuit-level model combining the Marcus-Hush-Chidsey electron current equation
and the Yakopcic equation for the state variable for describing resistive switching memory devices of
the structure metal–ionic conductor–metal. We extend the dynamics of the state variable originally
described by a first-order time derivative by introducing a fractional derivative with an arbitrary
order between zero and one. We show that the extended model fits with great fidelity the current-voltage characteristic data obtained on a Si electrochemical metallization memory device with Ag-Cu
alloy
\end{abstract}

\begin{multicols}{2}
[
\section{Introduction}
Substantial research efforts have been dedicated to the
development of electrically-controlled resistive switching in metal-insulator-metal (MIM) devices or memristors, going from new materials discovery1–7
to modelling
and simulation8–10, and design and applications3,11–13
.
With both memory and logic capabilities combined at
the hardware level, in addition to long retention times11
and high switching rates14 at relatively low energy
consumption1,15, these devices are favorably seen as the
next-generation building blocks for nonvolatile memories and neuromorphic computing applications11,12. In
a typical memristor, the resistive switching is based on
the electrically-stimulated change of cell resistance usually driven by internal ion redistribution, which actually depends not only on the applied excitation but also
on the past history of the excitation6
. Physical mechanisms associated with these reversible transitions have
been attributed to different effects including valencechange 16, electrochemical metallization17, and phase
change effects18. They can be either abrupt (binary)
or gradual (analogue), and evolve at different timescales,
leading to rich and complex device behaviors in this seemingly simple device structure of just three layers19. Furthermore, with the wide range of diversity in memristors
materials and their morphologies, operating mechanisms,
and manufacturing technologies there is an urgent need
for the development of a general model capable of capturing accurately and effectively their complex nonlinear
dynamics. This is crucial not only for the characterization and comparison between different memristor devices,
but also for the investigation of larger scale memristorbased circuits and hybrid hardware architectures, and
also to explore similar behaviors observed for instance in
biological synapse systems20
.
While models at different size scales and thus with different degrees of physical details and computational complexity have been developed for memritors, including but not limited to ab initio21, kinetic Monte Carlo, and finite
element method models22, in this work we focus on the
circuit-level (compact) current-voltage behavior of memristors. From this point of view, memristors are generally
described by the system of coupled equations23
\begin{eqnarray}
    i = G(v, x)v, \\
    x = f(x, v), 
\end{equation}
where i = i(t) is the current through the device, v = v(t)
is the applied voltage, and x = x(t) corresponds to a state
variable or a group of state variables that quantify the
internal dynamics of the device. These are, for example,
width of doping region, concentration of vacancies in the
gap region, and tunneling barrier width8
. State variables
can not be observed from external electrical behavior24
.
Eq. (1) follows the i-v curve of the resistive device in
question with G(v, x) being the generalized conductance,
whereas Eq. (2) describes the dynamics of its internal
state x based on its prehistory25. The actual state of a
memristor can only be determined by solving Eqs. (1)
and (2) self-consistently. Memristive systems as featured
in terms of Eqs. (1) and (2) are known to possess a
pinched hysteresis loop at the origin in the i-v plane in
the response to any periodic voltage source26
.
Being versatile and modular enough it is the Yakopcic model 27–29 which is most often used to simulate
the nonlinear i-v characteristic of wide range of memristors in response to sinusoidal and repetitive sweeping
inputs. The model takes into account electron transmission effects, voltage threshold for state variable motion,
and nonlinear velocity function for oxygen vacancies or
dopant drift, considered to be the most relevant internal
state information29. It follows on the steps of Strukov et
al. work30, and describes the memristor as two resistors
in series characterized by electron transmission equations
so that29
\begin{eqnarray}
    I(t) = h_1 (v)x + h_2 (v) (1-x).
\end{equation}
Here, h1 is used to model the behavior in the lowresistance state of the device, and h2 captures its behavarXiv:2302.09407v1 [cond-mat.mtrl-sci] 18 Feb 2023
2
ior in the high-resistance state. The two electron transmission equations are weighted and mixed by the state
variable x which is set to take values between zero and
one25. In memristive devices, it is the rate of change of
the state variable x that is explicitly determined (2), and
is given in the Yakopcic memristor model by the product
of the two composite functions g(v) and f(x) such that29:
]
\end{multicols}
\end{document}