\documentclass{article}
\usepackage[top=0cm, bottom=5cm, outer=2cm, inner=2cm, marginparwidth=5cm, marginparsep=5cm]{geometry}
\usepackage{multicol}
\usepackage{authblk}
\usepackage{blindtext}
\title{Fractional Marcus-Hush-Chidsey-Yakopcic current-voltage model for redox-based
resistive memory devices}

\author[1]{Georgii Paradezhenko}
\author[1]{Dmitrii Prodan}
\author[1]{Anastasiia Pervishko}
\author[1]{Dmitry Yudin}
\author[2]{Anis Allagui}
\affil[1]{Skolkovo Institute of Science and Technology, Moscow 121205, Russia}
\affil[2]{Department of Sustainable and Renewable Energy Engineering, University of Sharjah, Sharjah, P.O. Box 27272, United Arab Emirates}
\affil[3]{Department of Mechanical and Materials Engineering,
Florida International University, Miami, FL33174, United States}
\renewcommand{\abstractname}{}    % clear the title
\renewcommand{\absnamepos}{empty} % originally center
\begin{document}

\maketitle

\begin{abstract}
    We propose a circuit-level model combining the Marcus-Hush-Chidsey electron current equation
and the Yakopcic equation for the state variable for describing resistive switching memory devices of
the structure metal–ionic conductor–metal. We extend the dynamics of the state variable originally
described by a first-order time derivative by introducing a fractional derivative with an arbitrary
order between zero and one. We show that the extended model fits with great fidelity the currentvoltage characteristic data obtained on a Si electrochemical metallization memory device with Ag-Cu
alloy
\end{abstract}

\begin{multicols}{2}
[
\section{Introduction}
\blindtext \blindtext \blindtext \blindtext \blindtext \blindtext
]
\end{multicols}
\end{document}